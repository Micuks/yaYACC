通过这次语法分析程序的上手实践, 让我对编译原理课程的认识增加了除理论知识之外的内容, 
亲自动手实现语法分析程序也让我对编译器进行语法分析的过程有了更加深入的理解. \par

编译程序对源程序进行语法分析,
目的就是根据源语言的语法规则从源程序记号序列中识别出各种语法成分,
同时进行语法检查, 为语义分析和代码生成做准备. 语法分析工作由语法分析程序完成.
语法分析程序在编译程序模型中, 处于词法分析程序完成之后, 语义分析程序开始之前,
其输入是词法分析程序在扫描字符串源程序的过程中识别并生成的记号序列,
语法分析程序分析验证这个记号序列是不是符合该语言语法规则的一个程序, 如果是,
则输出其分析树; 如果不是, 则表明输入的记号序列中存在语法错误,
需要报告错误的性质和位置.\par

此外, 通过本次课程设计, 通过自己的动手亲身体验了将词法分析和语法分析等过程独立
处理的好处: 如可以将各部分需要实现的功能进行良好封装和解耦合, 对外只暴露接口和
提供服务, 各模块的具体实现对外部不可见, 简化了各部分实现的时候需要考虑的内容, 
从而在实现识别并去除空格, 注释等功能的时候思路更加清晰, 还可以让程序可移植性, 
可扩展性更强.\par

再次, 在这次课程设计中, 自己动手实现了LL(1)和LR(1)从任意给定文法到生成token流,
构建任意给定文法的LL(1)和LR(1)分析表, 并据此对token流进行分析,
对课本中介绍的各算法有了更加深入的理解, 反过来帮助了课程知识的学习.\par

通过自己实现一个简单的使用LL(1)或者LR(1)的yaYACC, 再对比对YACC的直接使用,
对YACC的背后原理有了更深认识的同时, 也体会到了YACC精妙的设计.\par

总体来说, 在本次课程设计过程中, 我对上学期所学形式语言和自动机知识, 以及本学期
所学的词法分析, 语法分析等内容都有了更加深刻的理解, 并掌握了运用方法; 此外, 编程能力, 程序
设计能力等也有了不小的提升.
