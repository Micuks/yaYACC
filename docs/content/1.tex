题目: 语法分析程序的设计与实现. 在基本的要求之外, 我拓展了程序的功能,
使其\textbf{可以读取任意的文法, 并可选生成LL(1)或者LR(1)分析表},
并对给定输入字符串进行对应的分析, 即LL(1)分析或LR(1)分析.\par

\subsection{LL(1)语法分析程序要求}
\subsubsection{必做功能要求}
\begin{enumerate}
	\item 编程实现算法4.2, 为给定文法自动构造预测分析表.
	\item 编程实现算法4.1, 构造LL(1)预测分析程序.
\end{enumerate}

\subsubsection{额外实现功能}
在作业要求的必做功能之外, LL(1)部分还实现了以下功能:
\begin{enumerate}
	\item 此程序能够\textbf{对任意文法进行分析}, 生成FIRST, FOLLOW集和LL(1)分析表,
	      并根据分析表对输入进行语法分析, 而不局限于作业所给算术表达式文法;
	\item 可以对文法进行消除左递归, 使CFG转化为复合LL(1)语法分析需求的文法;
	\item 在语法分析过程中能够动态检测并避免间接左递归;
	\item 输入语法支持正则表达式, 所有终结符均原生使用正则表达式进行存储,
	      并在分析输入的时候, 将输入按匹配正则表达式中最长的token进行切分.
\end{enumerate}

\subsection{LR(1)语法分析程序要求}
\subsubsection{必做功能要求}
\begin{enumerate}
	\item 构造识别给定文法所有活前缀的DFA;
	\item 构造该文法的LR分析表;
	\item 编程实现算法4.3, 构造LR分析程序;
\end{enumerate}

\subsubsection{额外实现功能}
\begin{enumerate}
	\item 此程序能够\textbf{对任意文法进行分析}, 生成FIRST, FOLLOW集,
	      并据此生成能够识别该文法所有活前缀的DFA, 以及LR(1)分析用到的分析表,
	      据此对输入字符串进行语法分析, 而不局限于作业所给算术表达式文法;
	\item 输入语法支持正则表达式, 所有终结符均原生使用正则表达式进行存储,
	      并且在分析输入的时候, 将输入按照匹配正则表达式中最长的token进行切分;
\end{enumerate}

\subsection{其他额外内容}
\begin{itemize}
	\item 按代码量3:2编写测试(实现代码量和测试代码量比例为3:2), 对每个类,
	      每个函数和每个操作都使用googletest设计了详细单元测试, 确保各部分功能正常;
\end{itemize}
