\subsection{实验题目}
题目: 语法分析程序的设计与实现.

\subsection{实验内容}
编写语法分析程序, 实现对算术表达式的语法分析. 在基本的要求之外,
我拓展了程序的功能, 使其\textbf{可以读取任意的文法,
	并可选生成LL(1)或者LR(1)分析表}, 并对给定输入字符串进行对应的分析,
即LL(1)分析或LR(1)分析.\par

作业中要求的另外两个方法也一并实现了, 即编写递归调用程序实现自顶向下的分析,
以及利用YACC自动生成语法分析程序, 调用LEX自动生成的语法分析程序.

\subsection{LL(1)语法分析程序}
\subsubsection{必做功能要求}
\begin{enumerate}
	\item 编程实现算法4.2, 为给定文法自动构造预测分析表.
	\item 编程实现算法4.1, 构造LL(1)预测分析程序.
\end{enumerate}

\subsubsection{额外实现功能}
在作业要求的必做功能之外, LL(1)部分还实现了以下功能:
\begin{enumerate}
	\item 此程序能够\textbf{对任意文法进行分析}, 生成FIRST, FOLLOW集和LL(1)分析表,
	      并根据分析表对输入进行语法分析, 而不局限于作业所给算术表达式文法;
	\item 可以对文法进行消除左递归, 使CFG转化为复合LL(1)语法分析需求的文法;
	\item 在语法分析过程中能够动态检测并避免间接左递归;
	\item 输入语法支持正则表达式, 所有终结符均原生使用正则表达式进行存储,
	      并在分析输入的时候, 将输入按匹配正则表达式中最长的token进行切分.
\end{enumerate}

\subsection{LR(1)语法分析程序}
\subsubsection{必做功能要求}
\begin{enumerate}
	\item 构造识别给定文法所有活前缀的DFA;
	\item 构造该文法的LR分析表;
	\item 编程实现算法4.3, 构造LR分析程序;
\end{enumerate}

\subsubsection{额外实现功能}
\begin{enumerate}
	\item 此程序能够\textbf{对任意文法进行分析}, 生成FIRST, FOLLOW集,
	      构造闭包(算法4.7), 构造LR(1)项目集规范族(算法4.8),
	      并据此生成能够识别该文法所有活前缀的DFA,
	      以及LR(1)分析用到的分析表(算法4.9),
	      据此对输入字符串进行语法分析, 而不局限于作业所给算术表达式文法;
	\item 输入语法支持正则表达式, 所有终结符均原生使用正则表达式进行存储,
	      并且在分析输入的时候, 将输入按照匹配正则表达式中最长的token进行切分;
\end{enumerate}

\subsection{其他额外实现功能}
\begin{enumerate}
	\item 按代码量2:1编写测试(实现代码量和测试代码量比例为2:1), 对每个类,
	      每个函数和每个操作都使用googletest设计了详细测试,
\end{enumerate}

作业中要求的另外两个方法也一并实现了.

\subsection{递归调用程序实现自顶向下的分析}
编写递归调用程序, 对符合语法规则的字符串能够接受并计算其结果;
如果不符合作业所给语法规则, 则拒绝.
