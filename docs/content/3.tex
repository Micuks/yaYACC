\subsection{语法分析程序的整体设计}
为了完成从解析所给任意文法到生成分析表,
并利用分析表分析给定输入字符串是否合法的任务, 将语法分析程序拆分设计如下:
\begin{itemize}
	\item 非终结符和终结符的定义, 操作和存储形式;
	\item 文法生成式以及LR1规约项目的定义, 操作和存储形式;
	\item LL1文法和LR1文法的数据成员, 方法成员和存储方式,
	      包括从文件读取语法并转化为适合的形式, 例如LL1文法进行消除左递归操作,
	      LR1文法需要进行拓广文法操作;
	\item 读取并依据上面得到的LL1或LR1文法处理字符串形式的输入,
	      获得作为下面LL1或LR1语法分析程序输入的token串; 为了提高输入文法的灵活性,
	      支持读取\textbf{正则表达式形式}的终结符, 且正则表达式是终结符的原生数据成员,
	      并应用在语法分析过程中;
	\item LL1和LR1语法分析程序的Parser, 包括FIRST, FOLLOW集的计算方法,
	      对LL1, 有分析表的构建和存储形式; 对LR1,
	      有能够识别所有活前缀的项目集DFA的构建和存储方法,
	      以及LR1语法分析表的构建和存储形式, 以及语法分析表中用到的规约,
	      移进和待约项目的表示形式等;
	\item googletest的测试的设计;
	\item CLI用户界面的设计;
\end{itemize}

对应源代码文件如下:

\begin{lstlisting}
  cli_parser.cpp
  cli_parser.hpp
  grammar.cpp
  grammar.hpp
  lex.cpp
  lex.hpp
  main.cpp
  main.hpp
  parser.cpp
  parser.hpp
  rule.cpp
  rule.hpp
  symbol.cpp
  symbol.hpp
\end{lstlisting}

\subsection{LL1语法分析程序}
\subsubsection{非终结符和终结符的设计}
定义在Symbol.hpp, 实现在Symbol.cpp中, 由于非终结符(Variable)和终结符(Terminal)共享许多共同点, 在此环节抽象来看,
唯一的不同就是终结符在我的实现中引入了正则表达式匹配功能. 因此,
使用共同的基类Symbol, 设计三个成员属性tag, index和identifier,
其中tag是variable和terminal中孤立的唯一数值标志,
而index是所有variable和terminal统一的唯一数值标志.
其中, Identifier是std::string类型, 存储符号的标识符.\par

得益于这种设计, 非终结符和终结符的长度不限, 且对大小写字母也没有限制.
对于终结符, 同时为了满足建立正则表达式pattern的需要, 均使用尖括号"<"和">"括起来.
涉及Variable和Terminal的同名不同操作的时候, 预留虚函数接口. 具体定义如下:
\begin{lstlisting}[language=c++]
class Symbol {

  public:
    Symbol(int tag, int index, string id)
        : tag(tag), index(index), identifier(id) {}

    const string getIdentifier() const { return identifier; }

    const int getTag() const { return tag; }

    const int getIndex() const { return index; }

    bool operator==(const Symbol &rhs) const {
        return (rhs.getIdentifier() == getIdentifier()) &&
               (rhs.getIndex() == getIndex()) && (rhs.getTag() == getTag());
    }

    friend ostream &operator<<(ostream &os, const Symbol &sym);

    enum SymbolType { variable, terminal };

    virtual SymbolType getType() = 0;

    virtual bool matcher(string token) = 0;

    virtual string toString() const;

  private:
    int tag;
    int index;
    string identifier;
};
\end{lstlisting}

对于非终结符(Variable),
与基类Symbol主要的差别是getType()函数返回的是SymbolType::variable,
即表示此符号的类型是非终结符. 具体定义如下:
\begin{lstlisting}[language=c++]
class Variable : public Symbol {
  public:
    Variable(int tag, int index, string id) : Symbol(tag, index, id) {}

    SymbolType getType() { return SymbolType(variable); }

    bool matcher(string token) { return -1; } // ?

    friend ostream &operator<<(ostream &os, const Variable &sym);
};
\end{lstlisting}

对于终结符(Terminal),
与基类Symbol的主要区别除了getType()函数的返回是SymbolType::terminal,
表示是终结符之外, 还有matcher()函数根据成员属性pattern进行正则表达式匹配,
具体定义如下:

\begin{lstlisting}[language=c++]
class Terminal : public Symbol {
  public:
    Terminal(int tag, int index, string id, regex pattern)
        : Symbol(tag, index, id), pattern(pattern) {}

    SymbolType getType() { return SymbolType(terminal); }

    bool matcher(string token) {
        if (token.length() == 0)
            return false;
        return (regex_match(token, pattern));
    }

    friend ostream &operator<<(ostream &os, const Terminal &sym);

  private:
    regex pattern;
};
\end{lstlisting}

\subsubsection{文法生成式的设计}
文法生成式, 又叫规则(Rule), 是文法中表示符号之间转换关系的生成式,
在本程序中定义在Rule.hpp中, 实现在Rule.cpp中, 分为基类Rule和继承类LR1Item,
其中后者被用在LR1文法的项目集DFA构建和存储中, 将在后面介绍.\par

Rule类主要有两个成员属性: Variable *lhs和std::vector<Symbol *> rhs.
对于一条语法规则S -> A <b>, 其中, lhs指->左边的S, 而在此处rhs中有两项,
分别是指向非终结符A和终结符b的指针. 具体定义如下:

\begin{lstlisting}[language=c++]
class Rule {
  public:
    Rule() : lhs(nullptr) {}
    Rule(Rule &rule) : Rule(rule.lhs, rule.rhs) {}
    Rule(const Rule &rule) : Rule(rule.lhs, rule.rhs) {}
    Rule(Variable *lhs, vector<Symbol *> rhs) : lhs(lhs), rhs(std::move(rhs)) {}

    bool leadToEpsilon(const Symbol *epsilon);

    void printRule();
    std::string toString();

    Variable *lhs;
    std::vector<Symbol *> rhs;
};

ostream &operator<<(ostream &os, const Rule &r);
bool operator==(Rule &ruleA, Rule &ruleB);
\end{lstlisting}

\subsubsection{LL(1)文法的设计}
文法的定义在Grammar.hpp中, 实现在Grammar.cpp中.
文法中的成员属性有开始符号startSymbol, 空符号epsilon, 栈底符号bos(指bottom of
stack), 还有非终结符集合variables, 终结符集合terminals, 规则集合rules.\par

在操作方面, 主要函数为获取非终结符v在左边的规则集合atLhsRules(),
以及获取符号s在右边的规则集合atRhsRules(), 根据tag获取符号的getSymbol(),
根据字符串进行正则匹配终结符的matchTerminal(),
判断非终结符是否有直接生成epsilon的toEpsilonDirectly(), 将CFG\footnote{CFG:
	Context free grammar, 上下文无关文法.}转化为符合LL1要求的文法的cfg2LL1(),
以及这一部分最重要的从文件加载文法的loadGrammar().具体定义如下:

\begin{lstlisting}[language=c++]
class Grammar {
  public:
    Grammar() : startSymbol(nullptr) {}
    Grammar(Grammar &g)
        : terminals((g.terminals)), variables((g.variables)), rules((g.rules)),
          startSymbol(g.startSymbol), epsilon(g.epsilon), bos(g.bos) {}
    vector<Terminal *> terminals;
    vector<Variable *> variables;
    vector<Rule> rules;

    Variable *startSymbol;
    Terminal *epsilon = new Terminal(-1, -1, "EPSILON", regex(""));
    Terminal *bos = new Terminal(-2, 0, "BOTTOM OF STACK", regex(""));

    vector<Rule> atLhsRules(Variable *v);
    vector<Rule> atRhsRules(Symbol *s);

    void loadGrammar(const char *filename);
    void loadGrammar(const std::string &filename);
    void printRules();
    Symbol *getSymbol(int tag);
    Terminal *matchTerminal(string str);
    bool toEpsilonDirectly(Variable *sym);

    void cfg2LL1();

  private:
    bool collectVariablesHaveLeftRecursion(
        std::unordered_set<Variable *> &variablesWhoseRulesHaveLeftRecursion);

    Variable *
    getNewVariable(std::unordered_set<Variable *> &newlyAddedVariables,
                   Variable *lhs, int &tagCnt, int &variablesIndexCnt);
};
\end{lstlisting}
其中, LL(1)文法通过继承已经构造好的文法Gramamr的方式构造:
\begin{lstlisting}[language=c++]
class LL1Grammar : public Grammar {
  public:
    LL1Grammar() : Grammar() {}
    // LL1 grammer need to eliminate left recursion.
    LL1Grammar(Grammar &g) : Grammar(g) { cfg2LL1(); }
};
\end{lstlisting}

\subsubsection{根据正则表达式将输入字符流转化为token流的Lex}
Lex根据给定文法g, 将输入字符串转化为终结符串, 提供给LL1Parser或者LR1Parser使用.
其主要成员变量为Grammar *g, 即其存储的文法. Lex定义在Lex.hpp中, 实现在Lex.cpp中,
具体定义如下:
\begin{lstlisting}[language=c++]
class Lex {
  public:
    Lex(Grammar *g = nullptr, bool v = false) : g(g), verbose(v) {}

    std::vector<Terminal *> *tokenize(const char *str);
    std::vector<Terminal *> *tokenize(std::string str);

    Grammar *g;

  private:
    bool verbose;
};
\end{lstlisting}

tokenize的过程主要使用函数如下:
\begin{lstlisting}[language=c++]
std::vector<Terminal *> *tokenize(const char *str);
\end{lstlisting}
其伪代码描述如下:
\begin{lstlisting}[language=c++]
std::vector<Terminal *> *Lex::tokenize(const char *rawStr) {
    auto tokens = new std::vector<Terminal *>;
    int idx = 0;
    char currCh = rawStr[idx++];
    std::vector<Terminal *> candidateMatches = g->terminals;
    candidateMatches.push_back(&空白字符);
    std::string token = "";

    // Match longest match
    while (currCh) {
        std::string tmp = token;
        tmp += currCh;
        int cntMatch = 0;

        for (auto &a : candidateMatches) {
            if (tmp满足终结符a的正则表达式) {
                cntMatch++;
            }
        }

        If matches, expand token. Else validate token and append it to
        tokens. If token is invalid, throw exception.

        // Process last token separately
        if (currCh == '\0') {
            Terminal *t = g->matchTerminal(token);
            if (t) {
                tokens->push_back(t);
            } else if (token不是空白字符串) {

                // Skip whitespace in inputStr. If there's any that remains
                // unmatched, it means that token is invalid. throw
                // exception.
            }
        }
    }

    // Append BOTTOM OF STACK symbol to token list.
    tokens->push_back(g->bos);

    return tokens;
}
\end{lstlisting}

\subsubsection{LL1语法分析表的构建}
LL1语法分析表的构造的定义在parser.hpp中, 实现在parser.cpp中. 构造语法分析表,
首先要构造first和follow集. 在Parser类中, 有成员属性grammar, 存储LL1文法,
以及parseTable, 存储分析表, 和lenParseTable, 记录分析表长度. 此外,
还有firstDict和followDict用于存储计算得到的first集和follow集.\par

在成员方法方面, 主要有first()对所给符号s计算FIRST(S),
并将计算过程中得到的所有first集合存储到firstDict中,
以及follow()对所给符号s计算FOLLOW(S),
并将计算过程中得到的所有first集合存储到firstDict中,
所有follow集合存储到followDict中.\par

对first和follow集合的计算有如下特点:
\begin{itemize}
	\item 对符号s只计算一次,
	      之后再调用s的first或follow集合的时候直接使用之前计算的结果;
	\item 仅当用到s的时候, s的first和follow集合才被计算, 减少了等待时间;
\end{itemize}
此外, 还有makeTable()成员函数, 用以计算LL(1)语法分析表.\par

Parser类的具体定义如下:
\begin{lstlisting}[language=c++]
class Parser {
  public:
    Parser() {}
    Parser(Grammar *g, bool verbose)
        : grammar(g), verbose(verbose), lenParseTable(0) {}

    ~Parser() {
        dropTable();

        for (auto &[k, v] : firstDict) {
            delete v;
        }

        for (auto &[k, v] : followDict) {
            delete v;
        }
    }

    void parse(std::vector<Terminal *> *tokens);

    void makeTable();
    void dropTable();

    void importFromFile(const char *filename);
    void exportToFile(const char *filename);

    void printFirstTable();
    void printFollowTable();

    std::string parseTableToString();
    void printParseTable();

  protected:
    std::unordered_set<Terminal *> first(Symbol *s);
    std::unordered_set<Terminal *> follow(Symbol *s);

    std::unordered_set<Terminal *> toResolveFollow(Symbol *rSym);
    std::unordered_set<Terminal *> &
    resolveFollow(Symbol *s, std::unordered_set<Terminal *> &followSet);

    std::unordered_map<Symbol *, std::unordered_set<Terminal *> *> firstDict;
    std::unordered_map<Symbol *, std::unordered_set<Terminal *> *> followDict;

    Grammar *grammar;
    bool verbose;

  private:
    std::string stackToString(std::stack<Symbol *> pda);
    std::string remainingTokensToString(std::vector<Terminal *>::iterator it,
                                        std::vector<Terminal *> *tokens);

    int **parseTable;
    int lenParseTable;
};
\end{lstlisting}

下面介绍LL1语法分析表构造中几个关键过程.
\subsubsection{FIRST集的构建}
FIRST集的构建使用如下函数:
\begin{lstlisting}[language=c++]
std::unordered_set<Terminal *> Parser::first(Symbol *s);
\end{lstlisting}

其伪代码描述如下:
\begin{lstlisting}[language=c++]
std::unordered_set<Terminal *> Parser::first(Symbol *s) {
    /**
     *
     * First(s) = {s} if s is terminal.
     *
     * If s is non-terminal,
     * if s -> epsilon, add epsilon to first(s);
     * for S -> s1s2...sn, if s1s2...s{i-1} -*> epsilon, add si to first(S).
     *
     */

    // First check if FIRST(s) has already been calculated.
    // If FIRST(s) is found in firstDict, return it directly.
    if (FIRST(S)已经计算过) {
        return 存储的FIRST(S);
    }

    // If not found, calculate it.
    std::unordered_set<Terminal *> firstSet;
    auto &epsilon = grammar->epsilon;

    if (s是终结符) {
        firstSet.insert((Terminal *)s);
    } else {
        // s is non-terminal
        vector<Rule> atLhsRules = s在左侧的规则;
        for (auto &r : atLhsRules) {

            // Check if r can go to epsilon
            Terminal *nonEpsilonTerminal = nullptr;
            Variable *nonEpsilonVariable = nullptr;
            for (r右侧每一个符号sym) {
                if (sym是终结符) {
                    if (sym == epsilon) {
                        continue;
                    } else {
                        nonEpsilonTerminal = (Terminal *)sym;
                        break;
                    }
                } else if (sym是非终结符) {
                    // First symbol at rule.rhs after epsilon is non-terminal.
                    if (grammar->toEpsilonDirectly((Variable *)sym)) {
                        continue;
                    } else {
                        // Add its First to FIRST(S).
                        nonEpsilonVariable = (Variable *)sym;
                    }
                }
            }
            if (nonEpsilonTerminal != nullptr) {
                // for S -> s1s2...sn, if s1s2...s_{i-1} -*> epsilon, add si
                // to first(S).
                将nonEpsilonTerminal添加到FIRST(S).
            } else if (nonEpsilonVariable != nullptr) {
                auto firstSetOfSym = first(nonEpsilonVariable);
                for (auto &a : firstSetOfSym) {
                    firstSet.insert(a);
                }
            } else {
                // If S -*> epsilon, add epsilon to first(S).
                将epsilon添加到FIRST(S).
            }
        }
    }

    // Put firstSet to firstDict, 记忆化存储FIRST(S), 避免重复计算.
    firstDict[s] = new std::unordered_set<Terminal *>(firstSet);
    return firstSet;
}
\end{lstlisting}
\subsubsection{FOLLOW集合的构建}
FOLLOW集合的构建使用如下函数:
\begin{lstlisting}[language=c++]
std::unordered_set<Terminal *> Parser::follow(Symbol *s);
\end{lstlisting}
其伪代码描述如下:
\begin{lstlisting}[language=c++]
/**
 *
 * Build FOLLOW(S) for symbol S.
 *
 * Add $ to FOLLOW(S) if S is start symbol of grammar;
 *
 * for each rule A -> ...S...,
 * if A -> aSB, then add FIRST(B) - {EPSILON} to FOLLOW(S).
 *
 * if A -> aSB where EPSILON in FIRST(B), then add FOLLOW(A) to FOLLOW(S).
 *
 * if A -> aS then add FOLLOW(A) to FOLLOW(S).
 *
 */
std::unordered_set<Terminal *> Parser::follow(Symbol *s) {

    // First check if FOLLOW(s) has already been calculated.
    // If FOLLOW(s) is found in followDict, return it directly.
    if (FOLLOW(S)已经被计算过) {
        return 已经计算的结果;
    }

    // If not found, calculate it.
    std::unordered_set<Terminal *> followSet;
    if (s是终结符) {
        FOLLOW(s) = BOTTOM_OF_STACK if s is terminal
        followSet.insert(grammar->bos);
    } else if (s是非终结符) {

        auto &epsilon = grammar->epsilon;
        Add $ to FOLLOW(S) if S is start symbol of grammar.
        if (s是开始符号) {
            followSet.insert(栈底符号);
        }

        auto atRhsRuleSet = s在右侧的规则;
        * for each rule A -> ...S...,
        for (auto &r : atRhsRuleSet) {
            auto posS = s在规则右边的位置;

             * if A -> aSB, then add FIRST(B) - {EPSILON} to FOLLOW(S).
            if (s不在最后一个) {
                 If FIRST(current symbol) contains EPSILON,
                     take next symbol into consideration.

                 * if A -> aSB where EPSILON in FIRST(B), then add
                 FOLLOW(A) to FOLLOW(S).

            } else if (s是最后一个) {
                if (s同时也是生成式左边的符号) {
                     the last symbol of current rule.rhs is the same as
                     current rule.lhs. Skip it to avoid endless loop.
                } else {

                     * if A -> aS then add FOLLOW(A) to FOLLOW(S).
                    }
                }
            }
        }
    } else {
        throw std::runtime_error(std::string("ERROR: Unknown Symbol Type"));
    }

     Resolve unresolved recursive follow() call.
    followSet = resolveFollow(s, followSet);

    followDict[s] = new std::unordered_set<Terminal *>(followSet);
    return followSet;
}
\end{lstlisting}
\subsubsection{LL(1)语法分析表的构造}
LL(1)语法分析表的构造使用如下函数:

\begin{lstlisting}[language=c++]
void Parser::makeTable();
\end{lstlisting}
其伪代码描述如下:
\begin{lstlisting}[language=c++]
void Parser::makeTable() {
     Drop parseTable if it already exists.

     Create new parseTable

    for (int i = 0; i < grammar->rules.size(); i++) {

         for each terminal in FIRST(a), add A->a... to M[A, t].

         If epsilon exists in FIRST(a), add A->a... to M[A, t] for each
         terminal in FOLLOW(A).
    }
}
\end{lstlisting}

\subsubsection{LL(1)语法分析过程的设计}
LL(1)语法分析过程使用函数如下:
\begin{lstlisting}[language=c++]
void Parser::parse(std::vector<Terminal *> *tokens);
\end{lstlisting}
其过程伪代码描述如下:
\begin{lstlisting}[language=c++]
void Parser::parse(std::vector<Terminal *> *tokens) {
    std::stack<Symbol *> pda = std::stack<Symbol *>();

    pda.push(g->bos);
    pda.push(g->startSymbol);

    while (pda不为空) {
        const Terminal *token = *it;
        Symbol *tos = pda.top(); // Top of stack

        // Try to match current terminal in token with symbol at the top of
        // stack.
        if (token是tos) {

            // tos matches token
            it++;
            currPos++;
            pda.pop();
        } else {
            // Tos failed to match, push symbols onto stack.
            if (token是terminal) {

                // Error occurred, unmatch terminal on top of stack
            } else {

                int ruleIdx = Get parseTable[variable][terminal].
                if (ruleIdx == -1) {
                    错误处理;
                }

                Rule r = g->rules[ruleIdx];
                pda.pop();
                if (规则r右边不直接有epsilon) {
                    将规则右边的符号逆序压入pda.
                }
            }
        }
    }

    std::cout << "INPUT ACCEPTED" << std::endl;
}
\end{lstlisting}

至此, LL1语法分析部分已经介绍完毕, 下面进行LR1语法分析部分的介绍:

\subsection{LR(1)语法分析程序的设计}
\subsubsection{LR(1)规约项目的设计}
LR(1)规约项目(LR1Item)在规则Rule的基础上, 添加了当前符号标识$\sim$,
以及向前看符号lookAhead. 因此, 在基类Rule的基础上,
派生类LR1Item增加了属性成员dotPos和lookAhead.  其类定义如下:
\begin{lstlisting}[language=c++]
class LR1Item : public Rule {
  public:
    LR1Item(Rule &rule, int dotPos, Terminal *lookAhead)
        : Rule(rule), dotPos(dotPos), lookAhead(lookAhead) {}
    LR1Item(LR1Item &old)
        : Rule(old.lhs, old.rhs), dotPos(old.dotPos), lookAhead(old.lookAhead) {
    }
    LR1Item(const LR1Item &old)
        : Rule(old.lhs, old.rhs), dotPos(old.dotPos), lookAhead(old.lookAhead) {
    }

    std::string toString() const;
    Rule getBase();

    int dotPos;
    Terminal *lookAhead;
};

ostream &operator<<(ostream &os, const LR1Item &item);
bool operator==(LR1Item &itemA, LR1Item &itemB);
bool operator==(const LR1Item &itemA, const LR1Item &itemB);
\end{lstlisting}

\subsubsection{LR(1)文法的设计}
LR(1)文法(LR1Grammar)在文法Grammar的基础上, 增加了拓广文法的操作. 类定义如下:
\begin{lstlisting}[language=c++]
class LR1Grammar : public Grammar {
  public:
    LR1Grammar() : Grammar() {}
    LR1Grammar(Grammar &g) : Grammar(g) { augmenting(); }

    Grammar *getBase();

  private:
    void augmenting();
};
\end{lstlisting}

\subsubsection{拓广文法}
拓广文法操作使用函数
\begin{lstlisting}[language=c++]
void LR1Grammar::augmenting();
\end{lstlisting}
主要操作为, 若原开始符号为S, 则新增符号S', 作为新的开始符, 并新增规则S'->S,
为LR1语法分析项目集DFA, 分析表的构建以及语法分析过程做准备.

\subsubsection{LR(1)语法分析}
LR(1)语法分析使用类LR1Parser, 基于基类Parser, 主要是改用了项目集DFA,
以及新的分析表格式. 其具体类定义如下:
\begin{lstlisting}[language=c++]
class LR1Parser : public Parser {
  public:
    LR1Parser() {}
    LR1Parser(Grammar *g, bool verbose)
        : Parser(LR1Grammar(*g).getBase(), verbose) {}
    // Augmenting via LR1Grammar constructor

    void parse(std::vector<Terminal *> *tokens);

    void printLR1ItemSets(std::ostream &os);
    void printLR1ParseTable(std::ostream &os);

    class ItemHasher {
      public:
        size_t operator()(const LR1Item &obj) const {
            static const std::hash<Symbol *> hasher;
            return (hasher(obj.lhs) << 16) + hasher(obj.rhs.front()) +
                   (obj.dotPos << 8);
        }
    };

    using ItemSet = std::unordered_set<LR1Item, ItemHasher>;

    std::vector<ItemSet> itemSets;

  private:
    enum class ActionType { REDUCE, SHIFT_GOTO };
    class Action {
      public:
        // Action type
        ActionType type;
        // Rule to use for reduce/accept
        Rule rule;
        // DFA state index
        int state;
    };

    void printItemSet(const ItemSet &itemSet, std::ostream &os);
    std::string lR1ItemSetToString(const ItemSet &itemSet);
    void getClosure(ItemSet &itemSet);
    ItemSet getGo(ItemSet &itemSet, Symbol *symbolToGo);

    int go(ItemSet &itemSet, Symbol *sym);

    class PairHasher {
      public:
        size_t operator()(const std::pair<int, Symbol *> &obj) const {
            // Use hash conflict to aggregate entries with the same state
            return std::hash<int>()(obj.first);
        }
    };

    std::unordered_map<std::pair<int, Symbol *>, Action, PairHasher> parseTable;
};
\end{lstlisting}

\subsubsection{LR(1)语法的项目集DFA的构建以及LR(1)语法的分析表的构建}
项目集DFA和LR(1)语法分析表的构建关键在于获取闭包的函数getClosure()的设计,
以及在计算goto的函数go()对项目集进行查找时更新. 之后只需进行如下的操作:
\begin{lstlisting}[language=c++]
    // Add starting item to initialItemSet
    auto initItemSet =
        ItemSet{LR1Item(grammar->atLhsRules(grammar->startSymbol).front(), 0,
                        grammar->epsilon)};

    // Get closure of initItemSet
    getClosure(initItemSet);

    itemSets.push_back(initItemSet);

    Rule emptyRule;

    // Loop until no change
    for (int index = 0; index < itemSets.size(); index++) {

        // Add shift/goto to parseTable
        for (每一个非终结符var) {
            auto goTo = go(itemSets[index], var);

            if (goTo存在) {
                parseTable.emplace(
                SHIFT_GOTO规则, goTO);
            }
        }

        // Add shift/goto to parseTable
        for (const auto &ter : 终结符) {
            auto goTo = go(itemSets[index], ter);

            if (goTo存在) {
                parseTable.emplace(
                    SHIFT_GOTO规则, goTo);
            }
        }

        // Add reduce to parseTable
        for (LR1Item item : itemSets[index]) {
            if (当前项目是归约项目) {
                parseTable.emplace(
                    REDUCE规则);
            }
        }
    }
\end{lstlisting}

下面对其中的关键函数go(), getClosure()进行介绍.
\subsubsection{getClosure()获取闭包函数}

该函数用于对给定项目集扩充为闭包, 定义如下:
\begin{lstlisting}[language=c++]
void LR1Parser::getClosure(ItemSet &itemSet);
\end{lstlisting}
, 具体伪代码如下:
\begin{lstlisting}[language=c++]
void LR1Parser::getClosure(ItemSet &itemSet) {
    bool isUpdated = true;

    // Repeat until no new item is added.
    while (isUpdated) {
        isUpdated = false;

        auto isT = [](Symbol *sym) {
            判断sym是否为终结符;
        };

        for (项目集itemSet中的每一个项目item) {
            // Reduction item
            if (item是归约项目) {
                // 对项目集闭包没有影响, 跳过;
                continue;
            }

            const auto &leader = item.rhs[item.dotPos];

            // Shift item
            if (leader是terminal) {
                continue;
            }

            // Reduction expecting item
            std::unordered_set<Terminal *> lookAheads;
            if (item.dotPos不是规则右侧最后一个符号前面) {
                // Leader is not the last symbol in the rhs.
                for (const auto &lookAhead : first(item.rhs[item.dotPos + 1])) {
                    lookAheads.insert(lookAhead);
                }
            } else {
                // Leader is the last symbol, use the same lookAhead.
                lookAheads.insert(item.lookAhead);
            }

            for (leader在左边的所有规则r) {
                for (const auto &lookAhead : lookAheads) {
                    LR1Item newItem(r, 0, lookAhead);

                    if (newItem不在已有项目集itemSet中) {
                        itemSet.insert(newItem);
                        isUpdated = true;
                    }
                }
            }
        }
    }
}
\end{lstlisting}

\subsubsection{go()函数介绍}
go()函数用于计算go(I, X), 即已有项目集I面临符号X的时候转移到的新状态项目集序号.
其函数定义如下:
\begin{lstlisting}[language=c++]
int LR1Parser::go(ItemSet &itemSet, Symbol *sym);
\end{lstlisting}

伪代码描述如下:
\begin{lstlisting}[language=c++]
int LR1Parser::go(ItemSet &itemSet, Symbol *sym) {
    auto newItemSet = getGo(itemSet, sym);

    if (newItemSet为空) {
        return -1;
    }

    if (newItemSet 已经存在于itemSets中) {
        return newItemSet在itemSets中的序号;
    }

    // go(I, X) is not in itemSets
    if (newItemSet不在itemSsets中) {
        itemSets.push_back(std::move(newItemSet));
    }

    // return index of newItemSet
    return i;
}
\end{lstlisting}

其中, getGo用于获取转移到的新项目集. 如果新项目集不合法, 则项目数量为0.
伪代码如下:
\begin{lstlisting}[language=c++]
LR1Parser::ItemSet LR1Parser::getGo(ItemSet &itemSet, Symbol *symbolToGo) {
    ItemSet newItemSet;
    for (itemSet中的项目item) {
        // Reduction item
        if (item是归约项目) {
            continue;
        }

        auto &leader = item.rhs[item.dotPos];
        // If leader is symbolToGo, insert it to newItemSet
        if (*leader == *symbolToGo) {
            LR1Item newItem(item);
            newItem.dotPos++;
            newItemSet.insert(newItem);
        }
    }

    // Get closure of newItemSet
    getClosure(newItemSet);

    return newItemSet;
}
\end{lstlisting}

\subsubsection{LR(1)语法分析过程的设计}
使用前面生成的LR(1)语法分析表, 配合符号栈(symbolStack),
状态栈(stateStack)对经过Lex处理后的token流进行分析, 并判断输入字符串是否被接受.
其伪代码如下:
\begin{lstlisting}[language=c++]
void LR1Parser::parse(std::vector<Terminal *> *tokens) {
    状态栈 stateStack;
    符号栈 symbolStack;

    // Init stateStack, 将状态0放入符号栈中
    stateStack.push_back(0);

    try {
        for (token in tokens) {
            auto state = stateStack.back();

            // Action
            LR1Parser::Action action;
            try {
                action = parseTable.at(state, token);

            } catch (...) {

                Error handler;
            }

            switch (action.type) {
            case LR1Parser::ActionType::SHIFT_GOTO:

                // Action
                stateStack.push_back(action.state);
                symbolStack.push_back(token);
                取下一个token;
                break;

            case LR1Parser::ActionType::REDUCE:
                if (action.rule.lhs == 开始符号) {
                    // Accept input, exit parsing...

                    // Action
                    pToken = tokens->end();
                } else {
                    auto lhsReducedTo = action.rule.lhs;

                    Action: reduce k symbols and their states

                    try {
                        action = parseTable.at(
                            std::make_pair(stateStack.back(), lhsReducedTo));
                    } catch (...) {

                        Error handler;
                    }

                    if (action.type != LR1Parser::ActionType::SHIFT_GOTO) {

                        Error handler;
                    }

                    // action.type is SHIFT_GOTO
                    stateStack.push_back(action.state);
                    symbolStack.push_back(lhsReducedTo);
                }
                break;
            default:

                Error handler;
                break;
            }
        }
    } catch (const std::exception &e) {

        Error handler;
    }

		Output parse result.
}
\end{lstlisting}

\subsection{GoogleTest 单元测试的设计}
由于此程序实现功能较多, 模块数量也多, 所以采用单元测试的方法进行调试,
既便利了debug, 也便利了观察输入本作业规定文法及其他文法的时候,
对字符串的分析过程.

GoogleTest是由Google发布的一个跨平台的(Linux, MacOS, Windows, Cygwin, Windows CE
and Symbian)的C++单元测试框架, 它提供了丰富的断言, 致命和非致命判断, 参数化,
"死亡测试"等.\par

在本程序的单元测试中, 主要使用了其对函数测试的TEST()和对类测试的TEST\_F()实体;
主要使用了如下语句:
\begin{itemize}
	\item EXPECT\_EQ(A, B). 判断A和B是否相等;
	\item EXPECT\_NE(A, B). 判断A和B是否不等;
	\item ASSERT\_EQ(A, B). 判断A和B是否相等. 如果不等, 不进行本单元中之后的测试;
	\item ASSERT\_NE(A, B). 判断A和B是否不等. 如果相等, 不进行本单元中之后的测试;
	\item EXPECT\_TRUE(str). 判断str是否为true;
	\item ASSERT\_TRUE(str). 判断str是否为true; 如果为false,
	      不进行本单元中后面的测试;
\end{itemize}

在本程序中, 对Symbol, Rule, Grammar, Parser, LR1Item,
LR1Grammar和LR1Parser均独立进行了测试, 分别在以下文件中:
\begin{itemize}
	\item test\_symbol.cpp,
	\item test\_rule.cpp,
	\item test\_grammar.cpp,
	\item test\_parser.cpp,
	\item test\_lr1\_grammar.cpp,
	\item test\_lrl1\_parser.cpp,
\end{itemize}
此外, 对用户界面功能也进行了测试, 同时进行了综合测试, 在test\_main.cpp中.

下面以LR1ParserTest为例进行展示, 由于具体内容较长, 故使用...进行省略:
首先定义测试类如下:
\begin{lstlisting}[language=c++]
class LR1ParserTest : public ::testing::Test {
  protected:
    void SetUp() {
        ...

        g3_ = new Grammar();
        g3_->loadGrammar("../grammars/g3.txt");
        p3_ = new Parser(g3_, true);
        lp3 = LR1Parser(g3_, true);
        l3_ = new Lex(g3_, true);
    }

    // There is memory leak for i didn't free allocated memory considering tight time.

    ...

    Grammar *g3_;
    Parser *p3_;
    LR1Parser lp3;
    Lex *l3_;

    std::string str2_1_ = std::string("423*384*23");
    std::string str2_2_ = std::string("(33+34)*45/32+8*(3*1+3)");
    std::string str2_21_ = std::string("(33+34)*45*32+8*(3*1+3)");
    std::string str2_3_ =
        std::string("(33+34)*45(32+8*(3*1+3)"); // Have grammar error that can't
                                                // pass LL(1) parse process.
    ...
};
\end{lstlisting}

根据测试类LR1ParserTest中的成员, 可以进行测试, 例如:
\begin{lstlisting}[language=c++]
TEST_F(LR1ParserTest, isGrammarLoadedCorrectly) {
    Grammar *g = lp1.grammar;
    g->printRules();

    EXPECT_EQ(g->startSymbol->getIdentifier(), "S'");

    g = lp2.grammar;
    g->printRules();
}
\end{lstlisting}

上面运行的TEST\_F()测试对测试类中定义的LR1语法分析器LR1Parser
lp1判断其加载的语法的起始符是否为所给文法的拓广文法起始符"S'"
来判断LR1Parser的语法加载功能是否正确.

具体的输出和测试将在下一节 程序测试部分 进行更详细的介绍.

\subsection{CLI用户界面的设计}

为了更好的对程序进行使用, 设计了CLI(Command Line Interface, 命令行用户界面),
让用户可以在终端中通过改变传递给本程序的参数的方式, 来使用本程序.
程序的帮助信息如下:

\begin{lstlisting}[language=c++]
"Options:"
"-h: show options"
"-g <filename>: generate a parser from the given grammar"
"-ll1 : Use LL(1) parser"
"-lr1 : Use LR(1) parser"
"-i <filename>: parse <filename>"
"-s \"<input string>\": parse given string"
"-v: give more output information"
""
"Examples:"
"ll(1) parse \"input string\" using parser_instance saved "
"previously:"
"    ./parser -p parser_instance -i \"input string\" -ll1"
"lr(1) parse input.txt using parser_instance saved previously:"
"    ./parse -lr1 -p parser_instance -i input.txt"
\end{lstlisting}

程序使用-g读取存储在文件中的语法, 可选使用-i参数来处理存储在文件中的字符串,
或者使用-s参数来处理作为参数传给程序的字符串;

\subsection{递归下降方法求算术表达式}
递归下降方法数域自顶向下分析方法. 自顶向下分析方法是一种面向目标的分析方法.
分析从文法的开始符号开始, 进行推导, 试图推出与输入符号串完全匹配的句子.
氛围非确定的和确定的两类, 不确定的分析方法是带回溯的分析方法,
实际上是一种穷举的试探方法, 效率低, 代价高, 一般极少采用.\par

递归下降方法是一种非确定的语法分析方法, 本质上是一种试探的过程,
即反复使用不同的产生式谋求匹配输入符号串的过程. 具体的伪代码如下:
\begin{lstlisting}[language=c++]
/**
 * Calculate the calue of an expression
 */
int RecursiveCalculation::expressionValue() {
    int result = 求所给表达式的一个项的值, 项可以是乘除法,
        可以是带括号的表达式, 可以是一个值;
    bool more = true;
    while (more) {
        if (字符是+或-) {
            std::cin.get();
            int value = termValue(); // 下一个表达式的值
            if (字符是+)
                result += value;
            else
                result -= value;
        } else
            more = false;
    }
    return result;
}

/**
 * calculate the value of a single item
 */
int RecursiveCalculation::termValue() {
    int result = factorValue();
    while (true) {
        char op = std::cin.peek();
        if (op是*或/) {
            std::cin.get();
            int value = factorValue(); // 下一个因子的值
            if (op是*)
                result *= value;
            else
                result /= value;
        } else
            break;
    }
    return result;
}

/**
 * Calculate the value of a single factor
 */
int RecursiveCalculation::factorValue() {
    int result = 0;
    char c = std::cin.peek();
    if (c是'(') { // 因子是一个表达式
        std::cin.get();
        result = expressionValue(); // 递归
        std::cin.get();
    } else { // 因子是个整数
        while (isdigit(c)) {
            result = 10 * result + c - '0';
            std::cin.get();
            c = 输入流的下一个字符;
        }
    }
    return result;
}
\end{lstlisting}

\subsection{YACC自动生成语法分析程序求算数表达式}
yacc (Yet Another Compiler Compiler)
是一个用来生成编译器的编译器(编译器代码生成器),
主要是一个用C语言生成的语法解析器, 采用LALR(1)语法分析方法.
具体代码如下, 表达式可换行, 识别";"作为表达式的结尾, 对合法的输入,
可以计算出表达式的值并接受; 对于不合法的表达式, 将会报错并拒绝. 具体代码如下:
\begin{lstlisting}[language=c++]
%{
#include<stdio.h>
#include<stdlib.h>
#include<ctype.h>

#define YYSTYPE double

int yylex();

extern int yyparse();

FILE* yyin;

void yyerror(const char* s);
%}

%token ADD SUB MUL DIV NUMBER   

%left ADD SUB

%left MUL DIV

%right UMINUS


%%
lines: lines expr ';' { printf("%g\n",$2); }
     | lines ';'
     |
     ;

expr: expr ADD expr { $$=$1+$3; }
    | expr SUB expr { $$=$1-$3; }
    | expr MUL expr { $$=$1*$3; }
    | expr DIV expr { $$=$1/$3; }
    | '(' expr ')'  { $$=$2; }
    | SUB expr %prec UMINUS { $$=-$2; }
    | NUMBER 
    ;
%%


int yylex() {
    int t;

    while(true) {

        t = getchar();

        if(t == ' ' || t == '\n') {
            ;
        } else if(isdigit(t)) {
            yylval=0;

            while (isdigit(t)) {
                yylval = 10 * yylval + t - '0';
                t = getchar();
            }

            ungetc(t,stdin);

            return NUMBER;
        } else {

            switch(t) {
            case '+':
                return ADD;
            case '-':
                return SUB;
            case '*':
                return MUL;
            case '/':
                return DIV;
            default:
                return t;
            }
        }
    }
}
   
int main(void) {
    yyin = stdin;

    do {
    yyparse();

    } while (!feof(yyin));

    return 0;
}

void yyerror(const char* s) {
    fprintf(stderr,"ERROR: %s\n",s);

    exit(1);
}
\end{lstlisting}
