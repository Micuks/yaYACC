\subsection{语法分析程序的设计}
为了完成从解析所给任意文法到生成分析表,
并利用分析表分析给定输入字符串是否合法的任务, 将语法分析程序拆分设计如下:
\begin{itemize}
	\item 非终结符和终结符的定义, 操作和存储形式;
	\item 文法生成式以及LR1规约项目的定义, 操作和存储形式;
	\item LL1文法和LR1文法的数据成员, 方法成员和存储方式,
	      包括从文件读取语法并转化为适合的形式, 例如LL1文法进行消除左递归操作,
	      LR1文法需要进行拓广文法操作;
	\item 读取并依据上面得到的LL1或LR1文法处理字符串形式的输入,
	      获得作为下面LL1或LR1语法分析程序输入的token串; 为了提高输入文法的灵活性,
	      支持读取\textbf{正则表达式形式}的终结符, 且正则表达式是终结符的原生数据成员,
	      并应用在语法分析过程中;
	\item LL1和LR1语法分析程序的Parser, 包括FIRST, FOLLOW集的计算方法,
	      对LL1, 有分析表的构建和存储形式; 对LR1,
	      有能够识别所有活前缀的项目集DFA的构建和存储方法,
	      以及LR1语法分析表的构建和存储形式, 以及语法分析表中用到的规约,
	      移进和待约项目的表示形式等;
	\item googletest的测试的设计;
	\item CLI用户界面的设计;
\end{itemize}

对应源代码文件如下:

\begin{lstlisting}
  cli_parser.cpp
  cli_parser.hpp
  grammar.cpp
  grammar.hpp
  lex.cpp
  lex.hpp
  main.cpp
  main.hpp
  parser.cpp
  parser.hpp
  rule.cpp
  rule.hpp
  symbol.cpp
  symbol.hpp
\end{lstlisting}

\subsection{非终结符和终结符的设计}
定义在Symbol.hpp, 实现在Symbol.cpp中, 由于非终结符(Variable)和终结符(Terminal)共享许多共同点, 在此环节抽象来看,
唯一的不同就是终结符在我的实现中引入了正则表达式匹配功能. 因此,
使用共同的基类Symbol, 设计三个成员属性tag, index和identifier,
其中tag是variable和terminal中孤立的唯一数值标志,
而index是所有variable和terminal统一的唯一数值标志.
其中, Identifier是std::string类型, 存储符号的标识符. 得益于这种设计,
非终结符和终结符的长度不限, 且对大小写字母也没有限制. 对于终结符,
同时为了满足建立正则表达式pattern的需要, 均使用尖括号"<"和">"括起来.
涉及Variable和Terminal的同名不同操作的时候, 预留虚函数接口. 具体定义如下:
\begin{lstlisting}[language=c++]
class Symbol {

  public:
    Symbol(int tag, int index, string id)
        : tag(tag), index(index), identifier(id) {}

    const string getIdentifier() const { return identifier; }

    const int getTag() const { return tag; }

    const int getIndex() const { return index; }

    bool operator==(const Symbol &rhs) const {
        return (rhs.getIdentifier() == getIdentifier()) &&
               (rhs.getIndex() == getIndex()) && (rhs.getTag() == getTag());
        // TODO: Implement !=
    }

    friend ostream &operator<<(ostream &os, const Symbol &sym);

    enum SymbolType { variable, terminal };

    virtual SymbolType getType() = 0;

    virtual bool matcher(string token) = 0;

    virtual string toString() const;

  private:
    int tag;
    int index;
    string identifier;
};
\end{lstlisting}

对于非终结符(Variable),
与基类Symbol主要的差别是getType()函数返回的是SymbolType::variable,
即表示此符号的类型是非终结符. 具体定义如下:
\begin{lstlisting}[language=c++]
class Variable : public Symbol {
  public:
    Variable(int tag, int index, string id) : Symbol(tag, index, id) {}

    SymbolType getType() { return SymbolType(variable); }

    bool matcher(string token) { return -1; } // ?

    friend ostream &operator<<(ostream &os, const Variable &sym);
};
\end{lstlisting}

对于终结符(Terminal),
与基类Symbol的主要区别除了getType()函数的返回是SymbolType::terminal,
表示是终结符之外, 还有matcher()函数根据成员属性pattern进行正则表达式匹配,
具体定义如下:

\begin{lstlisting}[language=c++]
class Terminal : public Symbol {
  public:
    Terminal(int tag, int index, string id, regex pattern)
        : Symbol(tag, index, id), pattern(pattern) {}

    SymbolType getType() { return SymbolType(terminal); }

    bool matcher(string token) {
        if (token.length() == 0)
            return false;
        return (regex_match(token, pattern));
    }

    friend ostream &operator<<(ostream &os, const Terminal &sym);

  private:
    regex pattern;
};
\end{lstlisting}

\subsection{文法生成式的设计}
文法生成式, 又叫规则(Rule), 是文法中表示符号之间转换关系的生成式,
在本程序中定义在Rule.hpp中, 实现在Rule.cpp中, 分为基类Rule和继承类LR1Item,
其中后者被用在LR1文法的项目集DFA构建和存储中, 将在后面介绍.\par

Rule类主要有两个成员属性: Variable *lhs和std::vector<Symbol *> rhs.
对于一条语法规则S -> A <b>, 其中, lhs指->左边的S, 而在此处rhs中有两项,
分别是指向非终结符A和终结符b的指针. 具体定义如下:

\begin{lstlisting}[language=c++]
class Rule {
  public:
    Rule() : lhs(nullptr) {}
    Rule(Rule &rule) : Rule(rule.lhs, rule.rhs) {}
    Rule(const Rule &rule) : Rule(rule.lhs, rule.rhs) {}
    Rule(Variable *lhs, vector<Symbol *> rhs) : lhs(lhs), rhs(std::move(rhs)) {}

    bool leadToEpsilon(const Symbol *epsilon);

    void printRule();
    std::string toString();

    Variable *lhs;
    std::vector<Symbol *> rhs;
};

ostream &operator<<(ostream &os, const Rule &r);
bool operator==(Rule &ruleA, Rule &ruleB);
\end{lstlisting}

\subsection{LL1文法的设计}
文法的定义在Grammar.hpp中, 实现在Grammar.cpp中.
文法中的成员属性有开始符号startSymbol, 空符号epsilon, 栈底符号bos(指bottom of
stack), 还有非终结符集合variables, 终结符集合terminals, 规则集合rules.
在操作方面, 主要函数为获取非终结符v在左边的规则集合atLhsRules(),
以及获取符号s在右边的规则集合atRhsRules(), 根据tag获取符号的getSymbol(),
根据字符串进行正则匹配终结符的matchTerminal(),
判断非终结符是否有直接生成epsilon的toEpsilonDirectly(), 将CFG\footnote{CFG:
	Context free grammar, 上下文无关文法.}转化为符合LL1要求的文法的cfg2LL1(),
以及这一部分最重要的从文件加载文法的loadGrammar().具体定义如下:

\begin{lstlisting}[language=c++]
class Grammar {
  public:
    Grammar() : startSymbol(nullptr) {}
    Grammar(Grammar &g)
        : terminals((g.terminals)), variables((g.variables)), rules((g.rules)),
          startSymbol(g.startSymbol), epsilon(g.epsilon), bos(g.bos) {}
    vector<Terminal *> terminals;
    vector<Variable *> variables;
    vector<Rule> rules;

    Variable *startSymbol;
    Terminal *epsilon = new Terminal(-1, -1, "EPSILON", regex(""));
    Terminal *bos = new Terminal(-2, 0, "BOTTOM OF STACK", regex(""));

    vector<Rule> atLhsRules(Variable *v);
    vector<Rule> atRhsRules(Symbol *s);

    void loadGrammar(const char *filename);
    void loadGrammar(const std::string &filename);
    void printRules();
    Symbol *getSymbol(int tag);
    Terminal *matchTerminal(string str);
    bool toEpsilonDirectly(Variable *sym);

    void cfg2LL1();

  private:
    bool collectVariablesHaveLeftRecursion(
        std::unordered_set<Variable *> &variablesWhoseRulesHaveLeftRecursion);

    Variable *
    getNewVariable(std::unordered_set<Variable *> &newlyAddedVariables,
                   Variable *lhs, int &tagCnt, int &variablesIndexCnt);
};
\end{lstlisting}
其中, LL1文法通过继承已经构造好的文法Gramamr的方式构造:
\begin{lstlisting}[language=c++]
class LL1Grammar : public Grammar {
  public:
    LL1Grammar() : Grammar() {}
    // LL1 grammer need to eliminate left recursion.
    LL1Grammar(Grammar &g) : Grammar(g) { cfg2LL1(); }
};
\end{lstlisting}

\subsection{根据正则表达式将输入字符流转化为token流的Lex}
Lex根据给定文法g, 将输入字符串转化为终结符串, 提供给LL1Parser或者LR1Parser使用.
其主要成员变量为Grammar *g, 即其存储的文法. Lex定义在Lex.hpp中, 实现在Lex.cpp中,
具体定义如下:
\begin{lstlisting}[language=c++]
class Lex {
  public:
    Lex(Grammar *g = nullptr, bool v = false) : g(g), verbose(v) {}

    std::vector<Terminal *> *tokenize(const char *str);
    std::vector<Terminal *> *tokenize(std::string str);

    Grammar *g;

  private:
    bool verbose;
};
\end{lstlisting}

\subsection{LL1语法分析表的构建}
LL1语法分析表的构造的定义在parser.hpp中, 实现在parser.cpp中. 构造语法分析表,
首先要构造first和follow集. 在Parser类中, 有成员属性grammar, 存储LL1文法,
以及parseTable, 存储分析表, 和lenParseTable, 记录分析表长度. 此外,
还有firstDict和followDict用于存储计算得到的first集和follow集. 在成员方法方面,
主要有first()对所给符号s计算FIRST(S),
并将计算过程中得到的所有first集合存储到firstDict中,
以及follow()对所给符号s计算FOLLOW(S),
并将计算过程中得到的所有first集合存储到firstDict中,
所有follow集合存储到followDict中. 对first和follow集合的计算有如下特点:
\begin{itemize}
	\item 对符号s只计算一次,
	      之后再调用s的first或follow集合的时候直接使用之前计算的结果;
	\item 仅当用到s的时候, s的first和follow集合才被计算, 减少了等待时间;
\end{itemize}
此外, 还有makeTable()成员函数, 用以计算LL1语法分析表.\par

Parser类的具体定义如下:
\begin{lstlisting}[language=c++]
class Parser {
  public:
    Parser() {}
    Parser(Grammar *g, bool verbose)
        : grammar(g), verbose(verbose), lenParseTable(0) {}

    ~Parser() {
        dropTable();

        for (auto &[k, v] : firstDict) {
            delete v;
        }

        for (auto &[k, v] : followDict) {
            delete v;
        }
    }

    void parse(std::vector<Terminal *> *tokens);

    void makeTable();
    void dropTable();

    void importFromFile(const char *filename);
    void exportToFile(const char *filename);

    void printFirstTable();
    void printFollowTable();

    std::string parseTableToString();
    void printParseTable();

  protected:
    std::unordered_set<Terminal *> first(Symbol *s);
    std::unordered_set<Terminal *> follow(Symbol *s);

    std::unordered_set<Terminal *> toResolveFollow(Symbol *rSym);
    std::unordered_set<Terminal *> &
    resolveFollow(Symbol *s, std::unordered_set<Terminal *> &followSet);

    std::unordered_map<Symbol *, std::unordered_set<Terminal *> *> firstDict;
    std::unordered_map<Symbol *, std::unordered_set<Terminal *> *> followDict;

    Grammar *grammar;
    bool verbose;

  private:
    std::string stackToString(std::stack<Symbol *> pda);
    std::string remainingTokensToString(std::vector<Terminal *>::iterator it,
                                        std::vector<Terminal *> *tokens);

    int **parseTable;
    int lenParseTable;
};
\end{lstlisting}

\subsection{LL1语法分析过程的设计}

\subsection{LR1规约项目的设计}
\subsection{LR1文法的设计}
\subsection{LR1语法的项目集DFA的构建}
\subsection{LR1语法的分析表的构建}
\subsection{LR1语法分析过程的设计}

\subsection{GoogleTest 单元测试的设计}
\subsection{CLI用户界面的设计}
