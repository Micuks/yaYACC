通过这次词法分析程序的上手实践, 让我对编译原理课程的认识增加了除理论知识之外的内容, 
亲自动手实现词法分析程序也让我对编译器进行词法分析的过程有了更加深入的理解. 

词法分析程序与语法分析程序之间的关系可以有3种, 分别是词法分析程序作为独立的
一遍, 词法分析程序作为语法分析程序的子程序, 和词法分析程序和语法分析程序作为
协同程序. 在本次课程设计中, 将词法分析程序作为了独立的一遍, 可以将词法分析程序
的输出放入到单独的中间文件, 让之后的语法分析程序读取中间文件即可获得词法分析
结果, 有利于提柜编译程序的效率.

此外, 通过本次课程设计, 通过自己的动手亲身体验了将词法分析和语法分析等过程独立
处理的好处: 如可以将各部分需要实现的功能进行良好封装和解耦合, 对外只暴露接口和
提供服务, 各模块的具体实现对外部不可见, 简化了各部分实现的时候需要考虑的内容, 
从而在实现识别并去除空格, 注释等功能的时候思路更加清晰, 还可以让程序可移植性, 
可扩展性更强.

再次, 在本次课程设计中我还尝试了利用FLEX自动生成词法分析程序, 在FLEX自动生成版本
和自己书写的版本的对比中, 体会到了FLEX功能的强大, 灵活和便利, 令我受益匪浅.

总体来说, 在本次课程设计过程中, 我对上学期所学形式语言和自动机知识, 以及本学期
所学的词法分析内容都有了更加深刻的理解, 并掌握了运用方法; 此外, 编程能力, 程序
设计能力等也有了不小的提升.
